\documentclass[10pt,]{article}
\usepackage[left=1in,top=1in,right=1in,bottom=1in]{geometry}
\newcommand*{\authorfont}{\fontfamily{phv}\selectfont}
\usepackage[]{mathpazo}


  \usepackage[T1]{fontenc}
  \usepackage[utf8]{inputenc}



\usepackage{abstract}
\renewcommand{\abstractname}{}    % clear the title
\renewcommand{\absnamepos}{empty} % originally center

\renewenvironment{abstract}
 {{%
    \setlength{\leftmargin}{0mm}
    \setlength{\rightmargin}{\leftmargin}%
  }%
  \relax}
 {\endlist}

\makeatletter
\def\@maketitle{%
  \newpage
%  \null
%  \vskip 2em%
%  \begin{center}%
  \let \footnote \thanks
    {\fontsize{18}{20}\selectfont\raggedright  \setlength{\parindent}{0pt} \@title \par}%
}
%\fi
\makeatother




\setcounter{secnumdepth}{0}



\title{Assesing Health Care Coverage and Access Utilizing the National Health
Interview Survey  }



\author{\Large John Diego\vspace{0.05in} \newline\normalsize\emph{University of Washington}   \and \Large Warren Wakuzawa\vspace{0.05in} \newline\normalsize\emph{University of Washington}   \and \Large Adrian Santiago\vspace{0.05in} \newline\normalsize\emph{University of Washington}  }


\date{}

\usepackage{titlesec}

\titleformat*{\section}{\normalsize\bfseries}
\titleformat*{\subsection}{\normalsize\itshape}
\titleformat*{\subsubsection}{\normalsize\itshape}
\titleformat*{\paragraph}{\normalsize\itshape}
\titleformat*{\subparagraph}{\normalsize\itshape}


\usepackage{natbib}
\bibliographystyle{apsr}



\newtheorem{hypothesis}{Hypothesis}
\usepackage{setspace}

\makeatletter
\@ifpackageloaded{hyperref}{}{%
\ifxetex
  \usepackage[setpagesize=false, % page size defined by xetex
              unicode=false, % unicode breaks when used with xetex
              xetex]{hyperref}
\else
  \usepackage[unicode=true]{hyperref}
\fi
}
\@ifpackageloaded{color}{
    \PassOptionsToPackage{usenames,dvipsnames}{color}
}{%
    \usepackage[usenames,dvipsnames]{color}
}
\makeatother
\hypersetup{breaklinks=true,
            bookmarks=true,
            pdfauthor={John Diego (University of Washington) and Warren Wakuzawa (University of Washington) and Adrian Santiago (University of Washington)},
             pdfkeywords = {healthcare, nhis, aca, medicaid, medicare},  
            pdftitle={Assesing Health Care Coverage and Access Utilizing the National Health
Interview Survey},
            colorlinks=true,
            citecolor=blue,
            urlcolor=blue,
            linkcolor=magenta,
            pdfborder={0 0 0}}
\urlstyle{same}  % don't use monospace font for urls



\begin{document}
	
% \pagenumbering{arabic}% resets `page` counter to 1 
%
% \maketitle

{% \usefont{T1}{pnc}{m}{n}
\setlength{\parindent}{0pt}
\thispagestyle{plain}
{\fontsize{18}{20}\selectfont\raggedright 
\maketitle  % title \par  

}

{
   \vskip 13.5pt\relax \normalsize\fontsize{11}{12} 
\textbf{\authorfont John Diego} \hskip 15pt \emph{\small University of Washington}   \par \textbf{\authorfont Warren Wakuzawa} \hskip 15pt \emph{\small University of Washington}   \par \textbf{\authorfont Adrian Santiago} \hskip 15pt \emph{\small University of Washington}   

}

}







\begin{abstract}

    \hbox{\vrule height .2pt width 39.14pc}

    \vskip 8.5pt % \small 

\noindent The National Health Interview Survey (NHIS) is the nation's largest
in-person household health survey. It has been conducted annually since
1957 by the National Center for Health Statistics (NCHS), which is a
part of the Centers for Disease Control and Prevention (CDC). A broad
range of topics are covered but we will specifically focus on healthcare
access and expenditure across a multitude of factors such as employer
information, race, and academic background.


\vskip 8.5pt \noindent \emph{Keywords}: healthcare, nhis, aca, medicaid, medicare \par

    \hbox{\vrule height .2pt width 39.14pc}



\end{abstract}


\vskip 6.5pt

\noindent  \section{INTRODUCTION}\label{introduction}

Healthcare\ldots{}(a hook, why we are doing it and then transition to
three questions we have).

The Patient Protection and Affordable Care Act (PPACA), commonly
shortened to the Affordable Care Act (ACA) and nicknamed Obamacare, is
one of the most important healthcare legislature, creating a significant
impact on the US Health care system. However, with the arrival of
President Trump in the Oval Office, and with the ACA repeal now
underway, some speculate huge consequences, such as increased death
rates(\citep{trump}).

The purpose of this paper is to assess healthcare coverage and access
across a multitude of factors. We will specifically look at three
aspects of health care - the first aspect is the relationship between
educational attainment and access to medical insurance, the second is an
overall look at the population on which had health care coverage, and
the final exploration covers healthcare expedinture over the years.

\section{RELATED WORK}\label{related-work}

A description of previous papers or projects related to your project.

\section{METHODS}\label{methods}

A detailed explanation of the techniques and algorithms you used to
solve the problem.

\section{RESULTS}\label{results}

The major insights resulting from your research.

\section{DISCUSSION}\label{discussion}

What are the implications of your results?

\section{FUTURE WORK}\label{future-work}

A description of how your system could be extended.
\newpage
\singlespacing 
\bibliography{/Users/elpresidente/Google Drive/INFO498C/INFO498Final/paper/master.bib}
\end{document}svm